\documentclass[gmd, manuscript]{copernicus} % uncomment to see what the 2 column final paper will look like.

\begin{document}

\title{Correcting a bias in a climate model with an augmented emulator}

% \Author[affil]{given_name}{surname}

\Author[1]{Doug}{McNeall}
\Author[2]{Jonny}{Williams}
\Author[1,3]{Richard}{Betts}
\Author[1]{Ben}{Booth}
\Author[3]{Peter}{Challenor}
\Author[1]{Peter}{Good}
\Author[1]{Andy}{Wiltshire}

\affil[1]{Met Office Hadley Centre, FitzRoy Road, Exeter, EX1 3PB, UK}
\affil[2]{NIWA, 301 Evans Bay Parade, Hataitai, Wellington 6021, New Zealand}
\affil[3]{University of Exeter, North Park Road, Exeter, EX4 4QE, UK}
%% The [] brackets identify the author with the corresponding affiliation. 1, 2, 3, etc. should be inserted.

\runningtitle{Correcting a bias in a climate model}

\runningauthor{Doug McNeall}

\correspondence{Doug McNeall (doug.mcneall@metoffice.gov.uk)}

\firstpage{1}

\maketitle

\begin{abstract}
A key challenge in developing flagship climate model configurations is the process of setting uncertain input parameters at values that lead to credible climate simulations. Setting these parameters traditionally relies heavily on insights from those involved in parameterisation of the underlying climate processes. Given the many degrees of freedom and computational expense involved in evaluating such a selection, this can be imperfect leaving open questions about whether any subsequent simulated biases result from mis-set parameters or wider structural model errors (such as missing or partially parameterised processes).  Here we present a complementary approach to identifying plausible climate model parameters, with a method of bias correcting subcomponents of a climate model using a Gaussian process emulator that allows credible values of model input parameters to be found even in the presence of a significant model bias.

A previous study \citep{mcneall2016impact} found that a climate model had to be run using land surface input parameter values from very different, almost non-overlapping parts of parameter space to satisfactorily simulate the Amazon and other forests respectively. As the forest fraction of modelled non-Amazon forests was broadly correct at the default parameter settings and the Amazon too low, that study suggested that the problem most likely lay in the model's treatment of non-plant processes in the Amazon region. This might be due to (1) modelling errors such as missing deep-rooting in the Amazon in the land surface component of the climate model, (2) a warm-dry bias in the Amazon climate of the model, or a combination of both.

In this study we bias correct the climate of the Amazon in a climate model using an ``augmented" Gaussian process emulator, where temperature and precipitation, variables usually regarded as model outputs, are treated as model inputs alongside regular land surface input parameters. A sensitivity analysis finds that the forest fraction is nearly as sensitive to climate variables as changes in its land surface parameter values. Bias correcting the climate in the Amazon region using the emulator corrects the forest fraction to tolerable levels in the Amazon at many candidates for land surface input parameter values, including the default ones, and increases the valid input space shared with the other forests. We need not invoke a structural model error in the land surface model, beyond having too dry and hot a climate in the Amazon region.

The augmented emulator allows bias correction of an ensemble of climate model runs and reduces the risk of choosing poor parameter values because of an error in a subcomponent of the model. We discuss the potential of the augmented emulator to act as a translational layer between model subcomponents, simplifying the process of model tuning when there are compensating errors, and helping model developers discover and prioritise model errors to target. 

\end{abstract}

\bibliographystyle{copernicus}
\bibliography{augmented.bib}

\end{document}
