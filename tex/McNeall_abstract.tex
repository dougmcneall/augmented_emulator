\documentclass[gmdd, manuscript]{copernicus} % uncomment to see what the 2 column final paper will look like.

\begin{document}

\begin{abstract}
We develop a method of bias correcting subcomponents of a climate model with a Gaussian process emulator, allowing credible values of model input parameters to be found even in the presence of a significant model bias. 

A previous study \citep{mcneall2016impact} found that a climate model had to be run using land surface input parameter values from very different, almost non-overlapping parts of parameter space in order to satisfactorily simulate the Amazon and other forests respectively. As the forest fraction of the other modelled forests was broadly correct at the default parameter settings and the Amazon too low, that study suggested that the problem most likely lay in the model's treatment of the Amazon region. This might be due to (1) modelling errors such as missing deep-rooting in the Amazon in the land surface component of the climate model, (2) a warm-dry bias in the Amazon climate of the model, or a combination of both.

In this study we bias correct the climate of the Amazon in a climate model using an ``augmented'' Gaussian process emulator, where temperature and precipitation, variables usually regarded as model outputs, are treated as model inputs alongside regular land surface input parameters. We explore the relationship between climate, land surface input parameters and forest fraction, finding that the forest fraction is nearly as sensitive to climate variables as any of the land surface inputs. Bias correcting the climate in the Amazon region using the emulator corrects the forest fraction to tolerable levels in the Amazon at many candidates for land surface input parameter values, including the default ones, and increases the valid input space shared with the other forests. We do not need to invoke a structural model error in the land surface model, beyond having too dry and hot a climate in the Amazon region.

Using the augmented emulator allows the bias correction a pre-existing coupled ensemble of climate model runs and reduces the risk of choosing poor parameter values because of an error in a subcomponent of the model. We discuss the potential of the augmented emulator to act as a translational layer between model subcomponents, simplifying the process of model tuning when there are potential compensating errors, and helping model developers prioritise model errors to target. Our technique has the potential to help choose good input parameters for a model, and to efficiently project the impacts of a changing climate, even when there are significant biases in a subcomponent of the model.

\end{abstract}


\end{document}